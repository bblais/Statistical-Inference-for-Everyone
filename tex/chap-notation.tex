\chapter{Notation and Standards}\label{app:notation}
%!TEX root = main.tex


\section{Useful Greek Letters}\label{sec:greek}
\begin{fullwidth}
\begin{tabular}{||l||l||} \hline\hline
\begin{tabular}{ccp{2in}}
$\alpha$& Alpha & slope of a line\\
$\beta$ & Beta & slope of a line, intercept\\
$\gamma$ &  Gamma & \\
$\Gamma$ & Gamma & \\
$\delta$ &  Delta & A small change in a variable\\
$\Delta$ & Delta & A change in a variable\\
$\epsilon$ &  Epsilon &\\
$\zeta$ &  Zeta & \\
$\eta$ &  Eta & \\
$\theta$ &  Theta & The parameters in a  binomial/beta distribution\\
$\Theta$ & Theta & \\
$\kappa$ & Kappa & \\
$\lambda$ &  Lambda & the mean in a poisson distribution \\
$\Lambda$ & Lambda & \\
$\mu$ &  Mu & the mean in a normal distribution (pronounced ``mew'')\\
$\nu$ &  Nu & (pronounced ``new'')\\
$\xi$ &  Xi & \\
$\Xi$ & Xi & 
\end{tabular}
&
\begin{tabular}{ccp{2in}}
$\pi$ & Pi & Represents the constant 3.1415$\cdots$, the ratio of the circumference to the diameter of a circle\\
$\Pi$ & Pi & A product of a series of numbers\\
$\rho$ &  Rho & \\
$\sigma$ &  Sigma & The standard width parameter of the normal distribution\\
$\Sigma$ & Sigma & A sum of a series of numbers \\
$\tau$ &  Tau & \\
$\phi$ &  Phi & \\
$\Phi$ & Phi & \\
$\chi$ &  Chi & A distribution related to the sum of normally distributed variables\\
$\psi$ &  Psi & \\
$\Psi$ & Psi & \\
$\omega$ &  Omega & \\
$\Omega$ & Omega & 
\end{tabular}\\ \hline\hline
\end{tabular}
\end{fullwidth}


\section{Some Math Notation}

\subsection{Variables}
A set of values, labeled with subscripts...
\beqn
x_{1}&=&1 \\
x_{2}&=&5 \\
x_{3}&=&-3 \\
x_{4}&=&2 \\
x_{5}&=&8
\eeqn
referred collectively as $x_{i}$.

\subsection{Sums}

\beqn
x_{1}+x_{2}+x_{3}+x_{4}+x_{5} = 1+5+(-3)+2+8 = 13
\eeqn

is equivalent to 

\beqn
\sum_{i=1}^{5} x_{i} = 1+5+(-3)+2+8 = 13
\eeqn
\subsection{Products}

\beqn
x_{1}\cdot x_{2}\cdot x_{3}\cdot x_{4}\cdot x_{5} = 1\cdot 5\cdot (-3)\cdot 2\cdot 8 = -240
\eeqn

is equivalent to 

\beqn
\prod_{i=1}^{5} x_{i} = 1\cdot 5\cdot (-3)\cdot 2\cdot 8 = -240
\eeqn


\subsection{Sample Mean}
The sample mean of a set of numbers is defined as...
\beqn
\bar{x} \equiv \frac{x_{1}+x_{2}+\cdots x_{N}}{N} 
\eeqn

In the example above

\beqn
\bar{x} \equiv \frac{x_{1}+x_{2}+x_{3}+x_{4}+x_{5}}{5} = 2\frac{3}{5}
\eeqn

It can also be written
\beqn
\bar{x} \equiv  \frac{\sum_{i=1}^{N} x_{i}}{N}
\eeqn
or
\beqn
\bar{x} \equiv  \frac{\sum_{i} x_{i}}{N}
\eeqn

\subsection{Sample Standard Deviation}

\beqn
s^{2}\equiv \frac{1}{N-1} \sum_{i=1}^{N} (x-\bar{x})^{2}
\eeqn
\marginnote{Although the justification for the $N-1$ part is beyond this book, one easy way to remember it is that the sample distribution of a set of numbers is an estimate for the $\sigma$ parameter of the normal distribution, representing the {\em spread} of the data.  You can think of the $N-1$ part as a check to keep you from doing the crazy thing of estimating a spread with only 1 data point!}
\beqn
s\equiv \sqrt{\frac{1}{N-1} \sum_{i=1}^{N} (x-\bar{x})^{2}}
\eeqn

\subsection{Estimates}

Any specific estimate of a parameter, such as $\theta$, is denoted with a hat, such as $\hat{\theta}$.


\subsection{Factorials}

Factorials are defined as
\beqn
N! = 1 \cdot 2 \cdot 3 \cdots (N-1) \cdot N
\eeqn
for example
\beqn
5! = 1 \cdot 2 \cdot 3\cdot 4 \cdot 5 = 120
\eeqn

The {\em N-choose-k} notation is a shorthand for the factorials that arise in binomial and Beta distributions.

\beqn
\nchoosek{N}{k} \equiv \frac{N!}{k!(N-k)!}
\eeqn


\section{Qualitative labels to probability values}

Rough guide for the conversion of qualitative labels to probability values used throughout the book.

\begin{center}
\begin{tabular}{||cc||} \hline\hline
term & probability \\\hline
virtually impossible & 1/1,000,000\\
extremely unlikely & 0.01 (i.e. 1/100) \\
very unlikely & 0.05 (i.e. 1/20) \\
unlikely & 0.2 (i.e. 1/5) \\
slightly unlikely & 0.4 (i.e. 2/5) \\
even odds & 0.5 (i.e. 50-50) \\
slightly likely & 0.6 (i.e. 3/5) \\
likely & 0.8 (i.e. 4/5) \\
very likely & 0.95 (i.e. 19/20) \\
extremely likely & 0.99 (i.e. 99/100) \\
virtually certain & 999,999/1,000,000\\ \hline\hline
\end{tabular}
\end{center}
