\DeclareSymbolFont{extraup}{U}{zavm}{m}{n}
\DeclareMathSymbol{\varheart}{\mathalpha}{extraup}{86}
\DeclareMathSymbol{\vardiamond}{\mathalpha}{extraup}{87}

\newcommand{\hearts}{\textcolor{red}{\varheart}}
\newcommand{\spades}{\spadesuit}
\newcommand{\diamonds}{\textcolor{red}{\vardiamond}}
\newcommand{\clubs}{\clubsuit}


%%
% Prints an asterisk that takes up no horizontal space.
% Useful in tabular environments.
\newcommand{\hangstar}{\makebox[0pt][l]{*}}

% Prints the month name (e.g., January) and the year (e.g., 2008)
\newcommand{\monthyear}{%
  \ifcase\month\or January\or February\or March\or April\or May\or June\or
  July\or August\or September\or October\or November\or
  December\fi\space\number\year
}

% Prints an epigraph and speaker in sans serif, all-caps type.
\newcommand{\openepigraph}[2]{%
  %\sffamily\fontsize{14}{16}\selectfont
  \begin{fullwidth}
  \sffamily\large
  \begin{doublespace}
  \noindent\allcaps{#1}\\% epigraph
  \noindent\allcaps{#2}% author
  \end{doublespace}
  \end{fullwidth}
}


% Inserts a blank page
\newcommand{\blankpage}{\newpage\hbox{}\thispagestyle{empty}\newpage}


% Typesets the font size, leading, and measure in the form of 10/12x26 pc.
\newcommand{\measure}[3]{#1/#2$\times$\unit[#3]{pc}}

% Macros for typesetting the documentation
\newcommand{\hlred}[1]{\textcolor{Maroon}{#1}}% prints in red
\newcommand{\hangleft}[1]{\makebox[0pt][r]{#1}}
\newcommand{\hairsp}{\hspace{1pt}}% hair space
\newcommand{\hquad}{\hskip0.5em\relax}% half quad space
\newcommand{\TODO}{\textcolor{red}{\bf TODO!}\xspace}
\newcommand{\ie}{\textit{i.\hairsp{}e.}\xspace}
\newcommand{\eg}{\textit{e.\hairsp{}g.}\xspace}
\newcommand{\na}{\quad--}% used in tables for N/A cells
\providecommand{\XeLaTeX}{X\lower.5ex\hbox{\kern-0.15em\reflectbox{E}}\kern-0.1em\LaTeX}
\newcommand{\tXeLaTeX}{\XeLaTeX\index{XeLaTeX@\protect\XeLaTeX}}
% \index{\texttt{\textbackslash xyz}@\hangleft{\texttt{\textbackslash}}\texttt{xyz}}
\newcommand{\tuftebs}{\symbol{'134}}% a backslash in tt type in OT1/T1
\newcommand{\doccmdnoindex}[2][]{\texttt{\tuftebs#2}}% command name -- adds backslash automatically (and doesn't add cmd to the index)
\newcommand{\doccmddef}[2][]{%
  \hlred{\texttt{\tuftebs#2}}\label{cmd:#2}%
  \ifthenelse{\isempty{#1}}%
    {% add the command to the index
      \index{#2 command@\protect\hangleft{\texttt{\tuftebs}}\texttt{#2}}% command name
    }%
    {% add the command and package to the index
      \index{#2 command@\protect\hangleft{\texttt{\tuftebs}}\texttt{#2} (\texttt{#1} package)}% command name
      \index{#1 package@\texttt{#1} package}\index{packages!#1@\texttt{#1}}% package name
    }%
}% command name -- adds backslash automatically
\newcommand{\doccmd}[2][]{%
  \texttt{\tuftebs#2}%
  \ifthenelse{\isempty{#1}}%
    {% add the command to the index
      \index{#2 command@\protect\hangleft{\texttt{\tuftebs}}\texttt{#2}}% command name
    }%
    {% add the command and package to the index
      \index{#2 command@\protect\hangleft{\texttt{\tuftebs}}\texttt{#2} (\texttt{#1} package)}% command name
      \index{#1 package@\texttt{#1} package}\index{packages!#1@\texttt{#1}}% package name
    }%
}% command name -- adds backslash automatically
\newcommand{\docopt}[1]{\ensuremath{\langle}\textrm{\textit{#1}}\ensuremath{\rangle}}% optional command argument
\newcommand{\docarg}[1]{\textrm{\textit{#1}}}% (required) command argument
\newenvironment{docspec}{\begin{quotation}\ttfamily\parskip0pt\parindent0pt\ignorespaces}{\end{quotation}}% command specification environment
\newcommand{\docenv}[1]{\texttt{#1}\index{#1 environment@\texttt{#1} environment}\index{environments!#1@\texttt{#1}}}% environment name
\newcommand{\docenvdef}[1]{\hlred{\texttt{#1}}\label{env:#1}\index{#1 environment@\texttt{#1} environment}\index{environments!#1@\texttt{#1}}}% environment name
\newcommand{\docpkg}[1]{\texttt{#1}\index{#1 package@\texttt{#1} package}\index{packages!#1@\texttt{#1}}}% package name
\newcommand{\doccls}[1]{\texttt{#1}}% document class name
\newcommand{\docclsopt}[1]{\texttt{#1}\index{#1 class option@\texttt{#1} class option}\index{class options!#1@\texttt{#1}}}% document class option name
\newcommand{\docclsoptdef}[1]{\hlred{\texttt{#1}}\label{clsopt:#1}\index{#1 class option@\texttt{#1} class option}\index{class options!#1@\texttt{#1}}}% document class option name defined
\newcommand{\docmsg}[2]{\bigskip\begin{fullwidth}\noindent\ttfamily#1\end{fullwidth}\medskip\par\noindent#2}
\newcommand{\docfilehook}[2]{\texttt{#1}\index{file hooks!#2}\index{#1@\texttt{#1}}}
\newcommand{\doccounter}[1]{\texttt{#1}\index{#1 counter@\texttt{#1} counter}}

\def\beqn{\begin{eqnarray*}}
\def\eeqn{\end{eqnarray*}}
\def\beq{\begin{eqnarray}}
\def\eeq{\end{eqnarray}}
\def\boxeqn{\begin{boxequation*}}
\def\eoxeqn{\end{boxequation*}}
\def\boxeq{\begin{boxequation}}
\def\eoxeq{\end{boxequation}}
\def\nn{\nonumber}
\def\bi{\begin{itemize}}
\def\ei{\end{itemize}}
\def\be{\begin{enumerate}}
\def\ee{\end{enumerate}}
\def\i{\item}
\newcommand{\cc}[1]{\begin{center}#1\end{center}}

\renewcommand{\P}[1]{P\left(\mbox{#1}\right)}
\newcommand{\Pg}[2]{P\left(\mbox{#1}|\mbox{#2}\right)}
\newcommand{\F}[1]{F\left(\mbox{#1}\right)}
\newcommand{\Fg}[2]{F\left(\mbox{#1}|\mbox{#2}\right)}

\newcommand{\prop}[2]{\mbox{#1}&\equiv& \left\{\parbox{2.4in}{#2}\right.}
\newcommand{\pquote}[1]{\begin{quote}{\it ``#1''}\end{quote}}
\newcommand{\psy}[2]{\includegraphics[height=#2]{#1}}
\newcommand{\psx}[2]{\includegraphics[width=#2]{#1}}
\newcommand{\image}[2]{\includegraphics{#1}}

\newcommand{\highlight}[3]{
\ \\
{\bf #1}\marginnote{{\bf #1} #3} #2 

}

\newcommand{\highlighttwo}[3]{
\ \\
{\bf #1}\marginnote{{\bf #1} #3} #2 

}

\makeatletter
\renewtheoremstyle{plain}%
{\item[\hskip\labelsep \theorem@headerfont ##1\ ##2\theorem@separator]}{\item[\hskip\labelsep \theorem@headerfont ##1\ ##2\theorem@separator]}

\makeatother
\newtheorem{definition}{Definition}[chapter]

\theoremstyle{plain}
\theoremheaderfont{\sc}
%\theoremindent0.5cm
%\theorembodyfont{\upshape}
\newtheorem{myexample}{Example}[chapter]

\newcommand{\example}[1]{
\begin{myexample}[#1]
#1
\end{myexample}
}

\newtheorem{myexercise}{Exercise}[chapter]

\newcommand{\exercise}[2]{
\begin{myexercise}[#1]
#2
\end{myexercise}
}

\newcommand{\getcite}{{\bf (GET CITATION) }}

\newcommand{\comment}[1]{}
\newcommand{\uncomment}[1]{#1}

\newcommand{\E}[1]{\cdot 10^{#1}}

\newenvironment{eqnwords}{
\begin{table}
\begin{center}
\begin{tabular}{c}
\toprule
\begin{minipage}{4in}
}{
\end{minipage}
\\
\bottomrule
\end{tabular}
\end{center}
\end{table}
}

\newcommand{\todo}[1]{\cc{$\left[ \parbox{3in}{\cc{\large \sc #1}} \right]$}}

\newcommand{\bvec}[1]{\mathbf{#1}}            % bold vector: \bvec{v}
\newcommand{\buvec}[1]{\mathbf{\hat{#1}}}     % bold unit vector
\newcommand{\uvec}[1]{\underline{#1}}         % underlined vector
\newcommand{\uuvec}[1]{\hat{\underline{#1}}}  % underlined unit vector
\newcommand{\zerobvec}{\mbox{\bf 0}}          % bold zero vector/matrix
\newcommand{\onebvec}{\mbox{\bf 1}}           % bold one vector/matrix
%
% 2 element column vector:
%                \twocvec{a}{b}  -->   ( a )
%                                      ( b )

\newcommand{\twocvec}[2]{\left(\begin{array}{c}
	#1 \\ #2
	\end{array}\right)}
\newcommand{\cvec}[1]{\left(\begin{array}{c}
	#1 
	\end{array}\right)}

\newcommand{\nchoosek}[2]{\twocvec{#1}{#2}}

%
% 2 element row vector:
%                \tworvec{a}{b}  -->   ( a  b )
\newcommand{\tworvec}[2]{\left(	{#1}\;\; {#2} \right)}
%
% indexed column vector:
%                \icvec{a}{1}{5}  -->   ( a_1 )
%                                       ( ... )
%                                       ( a_5 )
\newcommand{\icvec}[3]{\begin{array}{c}
	{#1}_{#2} \\ \vdots \\ {#1}_{#3}
	\end{array}}
%
% indexed row vector:
%                \irvec{a}{1}{5}  -->   ( a_1 ... a_5 )
\newcommand{\irvec}[3]{\begin{array}{ccc}
	{#1}_{#2} ,& \cdots &, {#1}_{#3}
	\end{array}}
%
% row vector:
%                \rvec{a}{1}{5}  -->   ( a_1 a_2 a_3 a_4 a_5 )
\newcount\m \newcount\n %
\def\rvec#1#2#3{ %
  \mbox{
    \m=#2 \n=#3%
    \loop $#1_{\number\m}$,
      \advance\m by 1 %
      \ifnum\m<\n %
    \repeat %
    $#1_{\number\m}$
  }
}
%
% ------ Matrix defs  ------
%
%
% diagonal matrix:
%      \diagmat{A}{B}  -->   ( A     0 )
%                            (   ...   )
%                            ( 0     B )
\newcommand{\diagmat}[2]{\left(\begin{array}{ccc}
#1 & & \mbox{\LARGE \bf 0} \\
 &\ddots &  \\
\mbox{\LARGE \bf 0} & &#2
\end{array}\right)}
%
% two by two  matrix:
%      \ttmat{a}{b}{c}{d}  -->   ( a b )
%                                ( c d )
\newcommand{\ttmat}[4]{\left(\begin{array}{cc}
#1 & #2 \\ #3 & #4
\end{array}\right)}


